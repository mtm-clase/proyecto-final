\documentclass[11pt,oneside,a4paper]{book}
\usepackage[%
  a4paper,%
  left = 20mm,%
  right = 20mm,%
  textwidth = 178mm,%
  top = 40mm,%
  bottom = 30mm,%
  %heightrounded,%
  headheight=70pt,%
  headsep=25pt,%
]{geometry}
\usepackage{graphicx}
%\usepackage[sfdefault,light]{Helvetica}
\usepackage[sfdefault,light]{FiraSans}
\usepackage{titlesec}
\usepackage{hyperref}
\hypersetup{
    colorlinks = true,
    allcolors  = link-blue, 
}
\usepackage{lastpage}
\usepackage{graphicx}
\usepackage{float}
\usepackage{xspace}
\usepackage{longtable}
\usepackage{tabularx}
\usepackage{color,colortbl}
\usepackage{longtable} % Paquete para tablas largas
\usepackage{colortbl}  % Colores en tablas
\usepackage[table,xcdraw]{xcolor} % Colores adicionales


\definecolor{link-blue}{RGB}{6,69,173}
\definecolor{dark-green}{RGB}{52,133,62}
\definecolor{light-blue}{RGB}{127,180,240}
\definecolor{dark-blue}{RGB}{72,120,224}
\definecolor{heading-grey}{RGB}{128,128,128}
\definecolor{heading2-grey}{RGB}{200,200,200}
\definecolor{Critical}{RGB}{192,0,0}
\definecolor{High}{RGB}{255,0,0}
\definecolor{Medium}{RGB}{255,192,0}
\definecolor{Low}{RGB}{255,255,0}
\definecolor{Informational}{RGB}{94,185,255}

\usepackage{listings}
\usepackage{enumitem}
\usepackage{array,booktabs}
\usepackage{fancyhdr}
\renewcommand{\footrulewidth}{0.2pt}
\renewcommand{\headrulewidth}{0.2pt}
\fancyfoot{}
\fancyhead{}
\fancyfoot[C]{Confidential}
\fancypagestyle{plain}{
    \fancyfoot[R]{\\ \textcolor{heading-grey}{\newline Página \thepage\ de \pageref{LastPage}}}
    \fancyfoot[C]{\textcolor{heading-grey}{\textbf{Proyecto de Fin de Ciclo -- \projectno}  }}
    \fancyhead[R]{\includegraphics[width=1cm]{img/logo.jpg}}
}
\fancypagestyle{fancy}{
    \fancyfoot[R]{\\ \textcolor{heading-grey}{\newline Página \thepage\ de \pageref{LastPage}}}
    \fancyfoot[C]{\textcolor{heading-grey}{\textbf{Proyecto de Fin de Ciclo -- \projectno}   }}
    \fancyhead{}
}
\thispagestyle{fancy}\pagestyle{plain}

\newcommand{\email}[1]{\href{mailto://#1}{#1}}
\newcommand{\newchapter}[1]{{\section*{#1}
\addcontentsline{toc}{chapter}{#1}}}
\newcommand{\newsection}[1]{{\subsection*{#1}
\addcontentsline{toc}{section}{#1}}}
\newcommand{\newsubsection}[1]{{\subsubsection*{#1}
\addcontentsline{toc}{subsection}{#1}}}
\usepackage[skip=10pt plus1pt, indent=0pt]{parskip}
\usepackage{float} % Añadir al preámbulo si no lo tienes

\usepackage{etoolbox}
\makeatletter
\patchcmd{\chapter}{\if@openright\cleardoublepage\else\clearpage\fi}{}{}{}
\makeatletter
\renewcommand\tableofcontents{%
    \if@twocolumn
      \@restonecoltrue\onecolumn
    \else
      \@restonecolfalse
    \fi
    \section*{\contentsname
        \@mkboth{%
           \MakeUppercase\contentsname}{\MakeUppercase\contentsname}}%
    \@starttoc{toc}%
    \if@restonecol\twocolumn\fi
    }
\makeatother


\titleformat{\section}
{\normalfont\huge\bfseries}{\thesection}{1em}{}
\titleformat{\subsection}
{\normalfont\Large\bfseries}{\thesubsection}{1em}{}
\titleformat{\subsubsection}
{\normalfont\large\bfseries}{\thesubsubsection}{1em}{}

% \titleformat{command}[shape]{format}{label}{sep}{before}[after]
% \titlespacing{command}{left spacing}{before spacing}{after spacing}[right]

\titlespacing{\section}{0pt}{1em}{0.5em}
\titlespacing{\subsection}{0pt}{0em}{0.25em}

\usepackage[T1]{fontenc}
\renewcommand*\oldstylenums[1]{{\firaoldstyle #1}}

\def\projectno{Cloud Privado Dinámico}


\usepackage[spanish]{babel}
\usepackage[utf8]{inputenc}
\usepackage[T1]{fontenc}

\begin{document}

\renewcommand{\headrulewidth}{0pt}

%%%%%%%%%%%%%%%%%%%%%%%%%%%%%%%%%%%%%%%%%
%%          Begin title page           %%
%%%%%%%%%%%%%%%%%%%%%%%%%%%%%%%%%%%%%%%%%

\begin{titlepage}
   \thispagestyle{fancy}
   \begin{center}
        \vspace*{8em}
   
        \centering\includegraphics[width=13cm]{img/logo_iescierva.jpg}

        \vspace{3em}

        \huge{\textbf{Proyecto Final de Ciclo \\
        Cloud Privado Dinámico}}

        \vspace{8em}

        \Large{Moisés Tamaalit Martínez\\
        2º Administración de Sistemas Informáticos en Red}

        \vspace{4em}

        \normalsize{
            IES Ingeniero de la Cierva\\
            Curso 2024/2025
        }

   \end{center}

    \vspace{2em}
    \normalsize{
        Fecha: \today \\
        Proyecto: \projectno \\
        Versión 1.0
    }
\end{titlepage}

\newpage

\tableofcontents

\newpage

%%%%%%%%%%%%%%%%%%%%%%%%%%%%%%%%%%%%%%%%%
%%           Begin contents            %%
%%%%%%%%%%%%%%%%%%%%%%%%%%%%%%%%%%%%%%%%%

\newchapter{Información de contacto}

\begin{table}[h]
\begin{center}
    \begin{tabular}{|m{5cm}|m{3cm}|m{7.5cm}|} % Anchos ajustados aquí
        \hline
        \rowcolor{heading-grey}\multicolumn{1}{|>{\centering\arraybackslash}m{5cm}|}{\textcolor{white}{\textbf{Nombre}}} & 
        \multicolumn{1}{>{\centering\arraybackslash}m{3cm}|}{\textcolor{white}{\textbf{Título}}} & 
        \multicolumn{1}{>{\centering\arraybackslash}m{7.5cm}|}{\textcolor{white}{\textbf{Contacto}}} \\ \hline
        Moisés Tamaalit Martínez & Alumno & Email: \href{mailto:moisestamaalit@gmail.com}{moisestamaalit@gmail.com} \\ \hline
    \end{tabular}
\end{center}
\end{table}


\newpage

\newchapter{Resumen}
Este proyecto presenta el desarrollo de una plataforma de \textit{hosting} VPS que permite desplegar y gestionar servidores virtuales de forma automatizada. La solución implementa una arquitectura de microservicios moderna, usando React, Node.js y Python, ofreciendo una interfaz intuitiva para usuarios sin conocimientos técnicos avanzados.

Los objetivos principales incluyen la automatización del despliegue de VPS, monitorización en tiempo real y gestión simplificada de servidores. Los resultados demuestran una plataforma funcional que cumple estos objetivos, proporcionando una solución segura y escalable.

\newpage

\newchapter{Introducción}

\newsection{Contextualización}
En la actualidad, la demanda de servidores virtuales privados (VPS) ha aumentado significativamente debido a la creciente necesidad de infraestructura \textit{cloud} flexible y escalable. Las empresas y desarrolladores buscan soluciones que les permitan desplegar y gestionar recursos de forma eficiente.

\newsection{Justificación}
La elección de este proyecto se basa en:
\begin{itemize}
    \item Experiencia previa en administración de sistemas
    \item Interés en tecnologías de virtualización
    \item Identificación de la necesidad de soluciones más accesibles
    \item Oportunidad de aplicar conocimientos del ciclo formativo
\end{itemize}


\newsection{Objetivos del Proyecto}
\textbf{Principales:}
\begin{itemize}
    \item Desarrollar una plataforma completa de gestión de VPS
    \item Automatizar el despliegue de servidores virtuales
    \item Implementar monitorización en tiempo real
\end{itemize}

\textbf{Secundarios:}
\begin{itemize}
    \item Crear una interfaz de usuario intuitiva
    \item Optimizar el rendimiento del sistema
    \item Documentar el proceso de desarrollo
\end{itemize}
\newpage
\newsection{Alcance}
\textbf{El proyecto incluye:}
\begin{itemize}
    \item Sistema de gestión de usuarios
    \item Panel de control de VPS
    \item Monitorización básica de recursos
    \item Automatización de despliegues
\end{itemize}

\textbf{No incluye:}
\begin{itemize}
    \item Sistema de facturación
    \item Soporte para múltiples centros de datos
    \item Backup automático de VPS
    \item Sistema de tickets de soporte
\end{itemize}

\newpage

\newchapter{Análisis previo}
\newsection{Necesidades detectadas}
El mercado actual de \textit{hosting} VPS presenta una barrera significativa para usuarios sin experiencia técnica avanzada. Los paneles de control existentes suelen ser complejos y poco intuitivos, dificultando la gestión de servidores virtuales. Además, la automatización del despliegue y configuración de VPS es limitada, requiriendo frecuentemente intervención manual.

\newsection{Requisitos técnicos y funcionales}
Los requisitos técnicos incluyen un servidor Proxmox VE como base de virtualización, mínimo 32GB de RAM y procesador con soporte para virtualización (Intel VT-x o AMD SVM). 

La plataforma debe soportar múltiples usuarios concurrentes y garantizar el aislamiento de recursos. 

Funcionalmente, se requiere una interfaz web amigable con el usuario, sistema de autenticación seguro, panel de control intuitivo y monitorización en tiempo real de recursos.

\newsection{Herramientas y tecnologías}
La implementación se basa en React con TypeScript para el \textit{frontend}, proporcionando una experiencia de usuario moderna y \textit{responsive}. El \textit{backend} utiliza Node.js con Express para la API principal, mientras que Python con FastAPI gestiona la interacción con Proxmox. MariaDB sirve como base de datos relacional, y Docker facilita la contenerización de servicios. Proxmox VE actúa como hipervisor base.

\newsection{Estudio de viabilidad}
El análisis de viabilidad indica que el proyecto es técnicamente factible con los recursos disponibles. Los costes se limitan principalmente al hardware necesario para el servidor Proxmox en este entorno. 

En un entorno de producción, hay que sumarle el coste de un bloque de IPs, los dominios que fueran a usarse y el mantenimiento del servidor. 

El tiempo estimado de desarrollo es de 4 meses, considerando las fases de diseño, implementación y pruebas. 

Los riesgos principales incluyen la complejidad de la integración con Proxmox y la necesidad de optimizar el rendimiento del sistema.

\newpage

\newchapter{Metodología}

\newsection{Frontend (React + TypeScript)}
Para el desarrollo del frontend se ha utilizado \textbf{React 19} con \textbf{TypeScript}, lo que permite crear una interfaz de usuario moderna y escalable. 

La ventaja de usar React es su capacidad para construir componentes reutilizables y su ecosistema robusto. TypeScript añade tipado estático, mejorando la mantenibilidad del código y reduciendo errores en tiempo de ejecución.

Se ha implementado \textbf{Tailwind CSS } para estilos y \textbf{Shadcn/ui} para componentes reutilizables. La navegación se gestiona con React Router y la autenticación se realiza mediante JWT.

Para desarrollar la interfaz web, dado que soy un administrador de sistemas y no un desarrollador web \textit{frontend}, he utilizado componentes predefinidos de Shadcn/ui, que proporcionan una base sólida y estilizada para la interfaz. Esto permite centrarme en la lógica de negocio y la integración con el \textit{backend} sin preocuparme por el diseño desde cero. Me he apoyado en desarrollo con modelos de inteligencia artificial para generar código y mejorar la eficiencia del desarrollo, como \textbf{Claude Sonnet 3.5} o \textbf{ChatGPT 4} usando GitHub Copilot.

\newsection{Backend (Node.js + Express)}
Para el desarrollo del \textit{backend} se ha utilizado \textbf{Node.js} con \textbf{Express}, lo que permite crear una API RESTful eficiente y escalable. Node.js es ideal para aplicaciones en tiempo real y Express facilita la creación de rutas y middleware.

\newsection{API Proxmox (Python + FastAPI)}
Para interactuar con el hipervisor, decidí hacer una API separada de la principal, utilizando \textbf{Python} con \textbf{FastAPI}. Esta elección se basa en securizar la interacción con el hipervisor, aprovechando la facilidad de uso de FastAPI para crear APIs RESTful y su excelente rendimiento. Además, Proxmox proporciona una API REST que se integra bien con FastAPI.

\newsection{Base de Datos (MariaDB)}
Escogí este sistema gestor de bases de datos (SGBD) por mi experiencia con este.

\newsection{Contenedorización}
Para el despliegue de la aplicación me decidí por usar contenedores Docker, lo que permite una fácil portabilidad y escalabilidad. Cada componente del sistema (frontend, backend, API Proxmox) se ejecuta en su propio contenedor, facilitando la gestión y el despliegue en diferentes entornos.

Me he apoyado con Docker Compose para orquestar los contenedores, definiendo servicios y redes necesarios para la aplicación. Esto permite un despliegue sencillo y reproducible en cualquier entorno compatible con Docker.

\newpage

\newchapter{Arquitectura del Sistema}

\newsection{Visión General}
El sistema se compone de varios componentes independientes que trabajan juntos:

\begin{itemize}
    \item Docker para contenerización
    \subitem Frontend web desarrollado con React y TypeScript
    \subitem Backend API en Node.js con Express
    \subitem API de Proxmox en Python con FastAPI
    \item Base de datos MariaDB
    \item Sistema de virtualización Proxmox VE
\end{itemize}

\newsection{Componentes}

\newsubsection{Frontend}
Desarrollado con React 19 y TypeScript, utiliza:
\begin{itemize}
    \item Tailwind CSS para estilos
    \item Shadcn/ui para componentes de interfaz
    \item React Router para navegación
    \item JWT para autenticación
\end{itemize}

\newsubsection{Backend API}
Implementado en Node.js con Express:
\begin{itemize}
    \item Express para el servidor web
    \item JWT para autenticación
    \item MariaDB para persistencia de datos
    \item Bcrypt para hash de contraseñas
\end{itemize}

\newsubsection{API de Proxmox}
Desarrollada en Python con FastAPI:
\begin{itemize}
    \item FastAPI para la API REST
    \item Proxmox API Client para interacción con Proxmox
    \item Pydantic para validación de datos
\end{itemize}

\newpage

\newchapter{Desarrollo del Proyecto}

\newsection{Implementación Frontend}
Descripción detallada del desarrollo frontend, incluyendo:
\begin{itemize}
    \item Estructura de componentes
    \item Sistema de autenticación
    \item Panel de control
    \item Monitorización
\end{itemize}

\newsection{Fase 3: Implementación Backend}
Desarrollo del servidor Express, incluyendo:
\begin{itemize}
    \item Sistema de rutas
    \item Middleware de autenticación
    \item Integración con base de datos
    \item API RESTful
\end{itemize}

\newsection{Fase 4: API Proxmox}
Implementación de la API de Proxmox:
\begin{itemize}
    \item Endpoints principales
    \item Sistema de plantillas
    \item Gestión de recursos
    \item Monitorización
\end{itemize}

\newpage

\newchapter{Resultados}

\newsection{Funcionalidades Implementadas}
\begin{itemize}
    \item Gestión de VPS
    \item Sistema de usuarios
    \item Monitorización
    \item Automatización
\end{itemize}

\newsection{Evaluación de Objetivos}
Análisis del cumplimiento de objetivos:
\begin{itemize}
    \item Funcionalidad
    \item Rendimiento
    \item Seguridad
    \item Usabilidad
\end{itemize}

\newpage

\newchapter{Seguridad}

\newsection{Autenticación y Autorización}
\begin{itemize}
    \item JWT para tokens de sesión
    \item Contraseñas hasheadas con bcrypt
    \item Validación de datos en endpoints
    \item CORS configurado por seguridad
\end{itemize}

\newsection{Seguridad en VPS}
\begin{itemize}
    \item Aislamiento de recursos
    \item Redes virtuales privadas
    \item Firewall configurado por defecto
    \item Actualización automática de plantillas
\end{itemize}

\newpage

\newchapter{Conclusiones y Trabajo Futuro}

\newsection{Objetivos Alcanzados}
\begin{itemize}
    \item Plataforma funcional de gestión de VPS
    \item Sistema de autenticación seguro
    \item Despliegue automatizado de servidores
    \item Monitorización en tiempo real
    \item Interfaz de usuario intuitiva
\end{itemize}

\newsection{Trabajo Futuro}
\begin{itemize}
    \item Sistema de copias de seguridad
    \item Métricas avanzadas
    \item Soporte para múltiples hipervisores
    \item API pública para integraciones
    \item Panel de administración avanzado
\end{itemize}

\newpage

\newchapter{Bibliografía}
\begin{itemize}
    \item Documentación oficial de React: https://reactjs.org/
    \item Documentación de FastAPI: https://fastapi.tiangolo.com/
    \item Documentación de Proxmox: https://pve.proxmox.com/wiki/
        \subitem - Documentación de la API de Proxmox: https://pve.proxmox.com/pve-docs/api-viewer/
    \item Documentación de Express: https://expressjs.com/
    \item Documentación de MariaDB: https://mariadb.org/documentation/
    \item Documentación de shadcn/ui: https://ui.shadcn.com/
    \item Documentación de Tailwind CSS: https://tailwindcss.com/docs
    \item Documentación de Docker: https://docs.docker.com/
\end{itemize}

\end{document}